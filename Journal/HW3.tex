\documentclass[11pt,twocolumn]{article}
\usepackage{fullpage}
\usepackage{amsmath}
\usepackage{url}

\title{Characteristics of Computer Science Journals}
\author{Shreya Reddy}

\begin{document}
\maketitle

\section{Journal 1:}

\begin{itemize}
   \item {\bf{Publisher}}: IEEE 
    \item {\bf{Name}}: Minimizing data and synchronization costs in one-way communication
     \item {\bf{url}}: \url{https://ieeexplore.ieee.org/document/895791}
      \item  The journal talks about the minimizing communication and synchronization costs to the performance potential of parallel computers. They discuss the technique, global data flow framework which optimizes communcation and synchronization into a one-way communication model. This model is a new paradigm that decouples message transmission and synchronization. The optimization technique used works with most of the general data alignments and distributions in languages like HPF. Researches have presented an approach based on data flow analysis combined with linear algebra to minimize synchronization and data transfer.

       \item {\bf{Name of Editor-in-Chief}}: H. Joel Trussell
        \item {\bf{Size of editorial board}}: 38
	\item {\bf{How frequently the journal is}}       

{\bf{published}}: The journal is published very frequently.
	 \item {\bf{Submission instructions for}}

{\bf{authors}}: Authors should submit the article as a PDF in double column, single spaced format. Papers can be up to 6 pages in length. A4 format needs to be used and US letter format is not accepted. Prior to submitting the paper, authors must add copyright notice at the bottom of the first page and should appear 0.5 cm to the left column. If the article was previously rejected after peer review, then a complete list of updates must be included in a separate document.
	  \item {\bf{Page limits for submitted articles}}: Most IEEE journals limit the number of pages per article. If the author chooses to exceed the page limit, they can pay for the overlength article charges.

	   \item {\bf{Double-blind reviewing policy}}: IEEE has a double-blind reviewing policy, where neither the author not the reviewers are aware of each others identity. But most IEEE publications use single-blind reviewing.

	    \item {\bf{Open access policy}}: IEEE has created IEEE OPEN, which is a publishing program that provides open access publishing. Hybrid Journals. Fully Open Topical Journals and Multidisciplinary journals are the three options that have open acess.

	\item {\bf{One additional piece of information about the journal that interests you}}: 
               \begin{enumerate}
                   \item Total Usage Since Feb 2011: 119
                    \item 3 citations across all articles \nocite{mobile}
               \end{enumerate}


\end{itemize}

\section{Journal 2:}

\begin{itemize}
   \item {\bf{Publisher}}: ACM New York, NY, USA
    \item {\bf{Name}}: ACM Transactions on Parallel Computing (TOPC)- Special Issue: Invited papers
     \item {\bf{url}}: \url{https://dl.acm.org/citation.cfm?id=3131272}
      \item In recent times, data streaming has become so popular that their is a continuous backup copying to a storage medium of data within a computer. It relies on continuous queries to process unbounded streams of data in the real-world. Data structures act as articulation points and maintain the state of data streaming, where it potentially supports high parallelism and balancing the work amongst them. Focusing on the problem of streaming multiway aggregration, a large volume of data are received from multiple streams.
       \item {\bf{Name of Editor-in-Chief}}: Phillip B. Gibbons
        \item {\bf{Size of editorial board}}: 16
	\item {\bf{How frequently the journal is }}

{\bf{published}}: The journal is published very frequently.
	 \item {\bf{Submission instructions for}}

{\bf{authors}}: ACM recommends the authors to make submissions in {\LaTeX} or Microsoft Word in the ACM style. The submissions should follow the ACM policies on prior publication, which includes correct citations of their work. It should also include a cover letter which documents where the submitted material has appeared before, ACM categories and subject descriptors should also be included. There is no page limit on the submission, but authors must ensure that the paper has an appropriate length for the material being presented. 

	  \item {\bf{Page limits for submitted}}

{\bf{ articles}}: 14 pages is the usual page limit for ACM submission papers in a two column format.

	   \item {\bf{Double - blind reviewing}}

{\bf{policy}}: AMC has a double-blind reviewing policy.

	    \item {\bf{Open access policy}} :ACM has an open access policy. The author has the option to manage the access to their work. ACM offers the author to purchase full right for his/her works through it's author-pays method.

	     \item {\bf{One additional piece of information about the journal that interests you}}: 
                        \begin{enumerate}
			 \item The high impact factor in ACM has a high rating in Computer Science publications.
                            \item  Downloads (12 Months): 117
			 \item Downloads (6 weeks): 9  \nocite{data}
		       \end{enumerate}


\section{3 Recently Published Articles from the Selected Journals}
\subsection{IEEE Articles}
     \begin{enumerate}
\item \underline{\bf{Advanced Code Generation for}}
\underline{\bf{High Performance Fortran}}
    
   \begin{itemize}
   \item The title seems a little broad for the average reader to comprehend what the topic is about. The title lacks the attention-grabbing factor that most of the good titles have. The title gives away very less information what the article is talking about.\cite{Fortran}
   \end{itemize}

\item \underline{\bf{Compilers: Principles}}
\underline{\bf{Techniques and Tools}}
   \begin{itemize}
    \item As mentioned above, this title is also too broad for the average reader to comprehend what the topic is about. \cite{compiler}
   \end{itemize} 

\item \underline{\bf{Interprocedural Data Flow}} 
\underline{\bf{Based Optimizations for}}
\underline{\bf{Distributed Memory Compilation}}
   \begin{itemize}
    \item The title clearly pictures what a great title should look like. It clearly references a topic that is not broad. The topic can be easily found for people who are researching on data flow or distributed systems.\cite{flow}
   \end{itemize}
 \end{enumerate}

\subsection{ACM Articles}

 	\begin{enumerate}
\item \underline{\bf{A Study of the Behavior}}
\underline{\bf{of Synchronization Methods in}}
\underline{\bf{Commonly Used Languages}}
\underline{\bf{and Systems}}
  \begin{itemize}
   \item The title is specific enough for its readers to comprehend what the article contains. Moreover, the reader can easily tell that they study the behavior of synchronization. It also states what languages they used in the system. \cite{sync}
  \end{itemize}

\item \underline{\bf{Three-Level Processing of}}
\underline{\bf{Multiple Aggregate Continuous}}
\underline{\bf{Queries}}
   \begin{itemize}
    \item The title isn't broad nor specific enough for the readers to understand it. Its a little confusing for the readers to know what exactly the article is all about. It just mentions its a three-level process for queries, due to its wordiness readers might find it confusing. But the author has conveyed a part of what the article talks about. \cite{query}
   \end{itemize}

\item \underline{\bf{Deterministic real-time analytics of }}
\underline{\bf{geospatial data streams through}}
\underline{\bf{ScaleGate objects}}
   \begin{itemize}
   \item This is another example of a perfect title. The readers can easily understand what the article is talking about. The author of the article could have titled it as "Deterministic real-time analytics of geospatial" but he/she has clearly mentioned what techonolgy they used to do the research. In this way it easily conveys the readers what the article is all about and how to research was done.\cite{real}
   \end{itemize}
 \end{enumerate}
	
\end{itemize}

\bibliographystyle{plain}
 \bibliography{refers.bib}


\end{document}